\def\DevnagVersion{2.13}% Assignment 2 for CMSC723
% Finite State Machines

\documentclass[11pt]{article}
\usepackage{latexsym}
\usepackage{listings}
\usepackage[usenames,dvipsnames]{color}
\usepackage[pdftex]{graphicx}
\usepackage{url}
\usepackage{hyperref}


%@modernhindi

% \date{}
\begin{document}
\begin{center}
{\Large{\textbf{ Natural Language Processing:  }}}\\
\mbox{}\\
{\Large{Assignment 2: Finite State Adventures\\Census Nomenclature \& Number Translation}}\\
% \mbox{}\\
% (\textsc{Not For Credit})\\
\mbox{}\\
{\large{Jordan Boyd-Graber}}\\
\mbox{}\\
{\large{Out: \textbf{25. August 2014}\\Due: \textbf{19. September 2014}}}\\
\end{center}


% \maketitle
\lstset{stringstyle=\ttfamily,language=Python,showstringspaces=False,tabsize=8,frameround=tttt,
		,keywordstyle=\color{Orange}\bfseries, stringstyle=\ttfamily\color{Green}
		,columns=fullflexible,identifierstyle=\ttfamily
		% , commentstyle=\itshape\color{Red}
}

\section*{Introduction} % (fold)
\label{sec:introduction}

This assignment is on finite state automata and will require you to
create automata in python. There are a total of three problems, and we
have provided some code to get you started, as well as a few utilities
that will make it easier for you to debug and test your code.  \\

We've provided an a \textsc{fst} implementation from \textsc{nltk}.
However, the documentation is less than perfect. Therefore, in this
section, we will give you a brief introduction to everything that you
will need to understand how to build finite state transducers in Python.
\\

To start building \textsc{fst}s, you need to first import the
\texttt{fst} module into your program's namespace. Then, you need to
instantiate an \texttt{FST} object. Once you have such an object, you
can start adding states and arcs to it. Listing~\ref{example} shows
how to build a very simple finite state transducer---one that removes
all vowels from any given word.

\begin{lstlisting}[float, label=example,caption=A 1-state transducer that deletes vowels, frame=trBL,escapechar=*, numbers=left, numberstyle=\tiny, numberblanklines=false]
# import the fst module
import fst

# import the string module
import string

# Define a list of all vowels for convenience
vowels = *\texttt{[}*'a', 'e', 'i', 'o', 'u'*\texttt{]}*

# Instantiate an FST object with some name
f = fst.FST('devowelizer')

# All we need is a single state ...
f.add_state('1')

# and this same state is the initial and the final state
f.initial_state = '1'
f.set_final('1')

# Now, we need to add an arc for each letter; if the letter is a vowel
# then then transition outputs nothing but otherwise it outputs the same
# letter that it consumed.
for letter in string.ascii_lowercase:
*\qquad*if letter in vowels:
*\qquad**\qquad*f.add_arc('1', '1', (letter), ())
*\qquad*else:
*\qquad**\qquad*f.add_arc('1', '1', (letter), (letter))

# Evaluate it on some example words
print ''.join(f.transduce(*\texttt{[}*'v', 'o', 'w', 'e', 'l'*\texttt{]}*))
print ''.join(f.transduce('e x c e p t i o n'.split()))
print ''.join(f.transduce('c o n s o n a n t'.split()))

\end{lstlisting}
\mbox{}\\
 Feel free to try out the example to see how it works on some of your
 own input. There are a few points worth mentioning:
\begin{enumerate}
	\item The Python \texttt{string} module comes with a few
          built-in strings that you might be able to use in this
          assignment for purposes of iteration as used in the example
          on line $23$. These are:
	\begin{itemize}
		\item \texttt{string.letters} : All letters, upppercased and lowercased
		\item \texttt{string.ascii\_lowercase} : All lowercased letters
		\item \texttt{string.ascii\_uppercase} : All uppercased letters
	\end{itemize}
	\item States can be added to an FST object by using its
          \texttt{add\_state()} method. This method takes a single
          argument: a unique string identifier for the state. Our
          example has only one state (line $13$). Furthermore, there
          can only be \textbf{one} initial state and this is indicated
          by assigning the state identifier to the FST object's
          \texttt{initial\_state} field (line $17$). However, there
          may be multiple final states in an FST. In fact, it is
          almost always necessary to have multiple final states when
          working with transducers. All final states may be so
          indicated by using the FST object's \texttt{set\_final()}
          method (line $18$).
	\item Arcs can be added between the states of an FST object by
          using its \texttt{add\_arc()} method. This method takes the
          following arguments (in order): the starting state, the
          ending state, the input symbol and, finally, the output
          symbol. If you wish to use single characters as input or
          output symbols, you must enclose them in parentheses (lines
          $25$ and $27$).  \\

	 However, if you wish to use entire words as input or output
         symbols, you must enclose the word in \textbf{square
           brackets} (not in parentheses). For example, if you wish to
         add an arc that takes the string \emph{ten} as input and
         returns the number string \emph{10} when going from state 1
         to 2, you should use:
	\begin{lstlisting}[label=foo1, frame=trBL,escapechar=*]
	f.add_arc('1', '2', *\texttt{[}*'ten'*\texttt{]}*, *\texttt{[}*'10'*\texttt{]}*)
	\end{lstlisting}
	 $\epsilon$'s may be indicated by an empty set of parentheses
         or square brackets, depending on the context. (line $25$).
	 \item An FST object can be evaluated against any input string
           by using its \texttt{transduce()} method. Here's how:
	\begin{enumerate}
		\item If your transducer uses \textbf{characters} as
                  input/output symbols, then the input to
                  \texttt{transduce()} must be a \textbf{list of
                    characters}. You may either directly input a list
                  of characters (line $30$) or you may convert a
                  string to a list of characters by spacing out its
                  characters and calling its \texttt{split()} method
                  (lines $31$ and $32$).
		\item If your transducer uses \textbf{words} as
                  input/output symbols, then the input to
                  \texttt{transduce()} should be a \textbf{list of
                    words}. Again, you can either explicitly use a
                  list of words or call the split method on a string
                  of words separated by whitespace. For example, say
                  your FST maps from strings like \emph{ten} and
                  \emph{twenty} to number strings \emph{10} and
                  \emph{20}, then to evaluate it on the input string
                  \emph{ten twenty}, you should use either:
		\begin{lstlisting}[label=foo2, frame=trBL,escapechar=*]
		f.transduce('ten twenty'.split())
		\end{lstlisting}
		\begin{center}{OR}\end{center}
		\begin{lstlisting}[label=foo3, frame=trBL,escapechar=*]
		f.transduce(*\texttt{[}*'ten', 'twenty'*\texttt{]}*)
		\end{lstlisting}
	\end{enumerate}
\end{enumerate}
% section introduction (end)

\section*{Provided Utilities}
To make it easier for you to solve the two programming problems below,
we have provided $3$ handy utilities in the included python file
\texttt{fsmutils.py}. These utility functions will help you to test
each transducer that you build and compose multiple transducers
together.  \\

 The first is \texttt{composechars()}, which allows you to compose any
 number of transducers (that use single characters as input strings)
 and evaluate it on any input string\footnote{Note that this function
   only performs composition in a practical sense and does not
   actually create a single composed transducer. However, for this
   assignment, the former is more than sufficient.}. For example, if
 you have created three transducers \texttt{f1}, \texttt{f2} and
 \texttt{f3} and you wish to evaluate their composition on the input
 string \texttt{S}, then you should use the following code:
	\begin{lstlisting}[label=composechars, frame=trBL,escapechar=*]
	from fsmutils import composechars
	output = composechars(S, f1, f2, f3)
	\end{lstlisting}
	The above function call computes (\texttt{f3} $\circ$
        \texttt{f2} $\circ$ \texttt{f1})(\texttt{S}). i.e., it will
        first apply transducer \texttt{f1} to the given input
        \texttt{S}, use the output of this transduction as input to
        transducer \texttt{f2} and so on and so forth. It will raise a
        generic exception if one or more input transducers do not work
        correctly. Note that since all transducers for this function
        use single characters as the input symbols, \texttt{S} must be
        a list of characters. \\

 The second utility function is \texttt{composewords()} which allows
 you to compose transducers that use words as input symbols, instead
 of single characters. The usage is similar to \texttt{composechars()}
 but the input string \texttt{S} must be a list of words in order to
 be used with this function. \\

 The final utility function is \texttt{trace()}. Given any single
 transducer \texttt{f} and a string \texttt{S}, this function will
 print the entire path taken through \texttt{f} when using \texttt{S}
 as the input. This can prove extremely invaluable for debugging any
 transducer. It may be used as follows:
	\begin{lstlisting}[label=trace, frame=trBL,escapechar=*]
	from fsmutils import trace
	trace(f, S)
	\end{lstlisting}

\section*{Problem 1: Soundex (35 points)} % (fold)
\label{sec:problem_2}
\textbf{Background}: The Soundex algorithm is a phonetic algorithm
commonly used by libraries and the Census Bureau to represent people's
names as they are pronounced in English. It has the advantage that
name variations with minor spelling differences will map to the same
representation, as long as they have the same pronunciation in
English. Here is how the algorithm works:
\begin{description}
	\item[Step 1:] Retain the first letter of the name. This may
          be uppercased or lowercased.
	\item[Step 2:] Remove all \textbf{non-initial} occurrences of
          the following letters: \texttt{a, e, h, i, o, u, w, y}. (To
          clarify, this step removes all occurrences of the given
          characters \emph{except} when they occur in the first
          position.)
	\item[Step 3:] Replace the remaining letters (except the
          first) with numbers:
	\begin{itemize}
		\item \texttt{b, f, p, v} $\rightarrow 1$
		\item \texttt{c, g, j, k, q, s, x, z} $\rightarrow 2$
		\item \texttt{d, t} $\rightarrow 3$
		\item \texttt{l} $\rightarrow 4$
		\item \texttt{m, n} $\rightarrow 5$
		\item \texttt{r} $\rightarrow 6$
	\end{itemize}
 	If two or more letters from the same number group were
        adjacent in the \emph{original} name, then \emph{only} replace
        the first of those letters with the corresponding number and
        ignore the others.
 	\item[Step 4:] If there are more than $3$ digits in the
          resulting output, then drop the extra ones.
 	\item[Step 5:] If there are less than $3$ digits, then pad at
          the end with the required number of trailing zeros.
\end{description}


 The final output of applying Soundex algorithm to any input string
 should be of the form \texttt{Letter Digit Digit
   Digit}. Table~\ref{tbl:soundex} shows the output of the Soundex
 algorithm for some example names.  \\

\begin{table}[htbp]
	\begin{center}
		\begin{tabular}{|c|c|}
			\hline
			\textbf{Input} & \textbf{Output} \\ \hline \hline
			Jurafsky & J$612$ \\ \hline
			Jarovski & J$612$ \\ \hline
			Resnik & R$252$ \\ \hline
			Reznick & R$252$ \\ \hline
			Euler & E$460$ \\ \hline
			Peterson & P$362$ \\ \hline
			\end{tabular}
		\caption{Example outputs for the Soundex algorithm.}\label{tbl:soundex}
\end{center}
\end{table}

 Construct an \textsc{fst} that implements the Soundex
 algorithm. Obviously, it is non-trivial to implement a single
 transducer for the entire algorithm. Therefore, the strategy we will
 adopt is a bottom-up one: implement multiple transducers, each
 performing a simpler task, and then compose them together to get the
 final output. One possibility is to partition the algorithm across
 three transducers:
\begin{enumerate}
	\item \textbf{Transducer 1}: Performs steps $1$-$3$ of the
          algorithm, i.e, retaining the first letter, removing letters
          and replacing letters with numbers.
	\item \textbf{Transducer 2}: Performs step $4$ of the
          algorithm, i.e., truncating extra digits.
	\item \textbf{Transducer 3}: Performs step $5$ of the
          algorithm, i.e., padding with zeros if required. \\
\end{enumerate}

 Note that each of these three transducers will have characters as
 input/output symbols.  \\

 To make things easier for you, we have provided the file
 \texttt{soundex.py} which is where you will write your code. It
 already imports all needed modules and functions (including
 \texttt{fsmutils.py}). It also creates three transducer objects---as
 dictated by the bottom-up strategy outlined above---such that all you
 should have to do is to figure out the states and arcs required by
 each transducer. It also contains code that allows you to input a
 single name on the command line to get the output.

\noindent \textbf{Notes}:
\begin{enumerate}
	\item[(i)] Your grade will only be evaluated on the command
          line.  You do not have to factor the code in the same way as
          the unit tests; however, the unit tests will no longer
          work.

          \item[(ii)] While we have provided you with sample unit
          tests containing some names, it might be very useful to test
          your code on other names. For comparison purposes, you may
          use one of the many Soundex calculators available online.
\end{enumerate}
% section problem_2 (end)

\section*{Problem 2 (55 points)} % (fold)
\label{sec:problem_3}
\textbf{Background}: In the French language, Arabic numerals that we
use in everyday can be spelled out just like they can in English. For
example, the numeral 175 is written as \texttt{one hundred seventy
  five} in English and \textit{cent soixante quinze} in French.

French is interesting becuase they have a mixture of a deximal (base
10) and vegesimal (base 20) system, created by committee to placate
two different regions of Franch that used different systems.  Thus,
for example, you have: \\
\begin{tabular}{|l|l|l|}
\hline
Arabic & French & English Gloss \\
\hline
4 & quatre & four \\
6 & six & six \\
16 & seize & sixteen \\
20 & vingt & twenty \\
26 & vingt six & twenty six \\
56 & cinquante six & fifty six \\
66 & soixante six & sixty six \\
76 & soixante seize & sixty sixteen  \\
80 & quatre vingt & four twenty \\
86 & quatre vingt six & four twenty six \\
96 & quatre ving seize & four twenty sixteen \\
996 & neuf cent quatre ving seize & nine hundred four twenty sixteen
\\
\hline
\end{tabular}

\noindent To summarize:
\begin{itemize}
\item Between 70--79 and 90--99, French numbers use a
vegesimal base system.  For everything else, it is base 10.
\item Numbers congruent to 1 mod 10 in the decimal reange have an ``and''.  For example, 21
  is \textit{vingt et un} (``twenty and one'')
\item Numbers larger than 100 are written as $x$ hundred.  For
  example, 600 becomes \textit{six cent} (``six hundred'').
\item To make things simpler, we'll say that 80 is \textit{quatre vingt} (without an ``s'').
\end{itemize}

You may want to consult an online reference, such
as \url{https://www.udemy.com/blog/french-numbers-1-1000/}.

Construct an \textsc{fst} in \textsc{nltk} that can translate any
given Arabic numeral into its corresponding French string. For the
sake of convenience, you will only be given integer input less than
1000.


\noindent \textbf{Notes}:
\begin{itemize}
	\item[(i)]  Just like for the Soundex problem, we have provided a file
          called \texttt{french\_count.py} with boilerplate code. You
          should add your code to this file, having the
          \texttt{french\_count} function return a transducer that
          produces the correct output.
	\item[(ii)] You should not type any French string into this
          file.  All necessary French strings are in a dictionary
          called \texttt{kFRENCH\_TRANS}.  Adding additional French
          strings will cause you to lose points.
\end{itemize}
% section problem_3 (end)

\section*{English Morphology (10 points)}

English is one of the least interesting languages morphologically, but
it's a good warm up (Chinese, Vietnamese are even less
interesting). If you take a look at \texttt{tests.py}, you can see
some of the words we're going to working with: pack, ice, frolic,
pace, ace, traffic, lilac, lick.  Our goal is to be able to generate
things like "spruced" (from "spruce+d") and "picnicking" (from
"picnic+ed") using regular expressions (which are magically tranformed
into finite state machines\footnote{How this happens is outside the
  scope of the course.}).

The shell for this part of the assignment is in
\texttt{morphology.py}. Essentially all you need to do is implement an
additional regular expression rule that will correctly handle c/ck
alternations. Before doing this, you should be able to execute
\texttt{tests.py} to see the tests failing.



\section*{Turning in Your Assignment}

As usual, submit your completed code to
\href{https://moodle.cs.colorado.edu/course/view.php?id=161}{Moodle}.
Make sure:
\begin{itemize}
  \item You do not change filenames or function names
  \item Turn in all the files you modified individually (do not zip or
    combine them)
  \item You do not change the \textsc{api} of the functions
  \item Do not add \texttt{print} statements to the code you turn in
  \item Add your name as a comment to all files you turn in
\end{itemize}


\end{document}
