\def\DevnagVersion{2.13}% Assignment 2 for CMSC723
% Finite State Machines

\documentclass[11pt]{article}
\usepackage{latexsym}
\usepackage{listings}
\usepackage{hyperref}
\usepackage[usenames,dvipsnames]{color}
\usepackage[pdftex]{graphicx}


%@modernhindi

\hypersetup{colorlinks=true,linkcolor=blue}

% \date{}
\begin{document}
\begin{center}
{\Large{\textbf{ Natural Language Processing:  }}}\\
\mbox{}\\
{\Large{Assignment 1: There once was a Python warmup from \dots}}\\
% \mbox{}\\
% (\textsc{Not For Credit})\\
\mbox{}\\
{\large{Jordan Boyd-Graber}}\\
\mbox{}\\
{\large{Out: \textbf{25. August 2014}\\Due: \textbf{12. September 2014}}}\\
\end{center}


% \maketitle
\lstset{stringstyle=\ttfamily,language=Python,showstringspaces=False,tabsize=8,frameround=tttt,
		,keywordstyle=\color{Orange}\bfseries, stringstyle=\ttfamily\color{Green}
		,columns=fullflexible,identifierstyle=\ttfamily
		% , commentstyle=\itshape\color{Red}
}

\section*{Introduction} % (fold)
\label{sec:introduction}
First, check out the Github repository with the course homework templates:

\url{git://github.com/ezubaric/cl1-hw.git}

The goal of this assignment is to create a piece of code that will
determine whether a poem is a
\href{http://en.wikipedia.org/wiki/Limerick_(poetry)}{limerick} or
not.  To do this, we will be using the
\href{http://www.speech.cs.cmu.edu/cgi-bin/cmudict}{CMU pronunciation
  dictionary} (which is covered in the \href{http://www.nltk.org/book/ch02.html}{second chapter of the NLTK book}).

A limerick is defined as a poem with the form AABBA, where the A lines
rhyme with each other, the B lines rhyme with each other (and not the
A lines).  (English professors may disagree with this definition, but
that's what we're using here to keep it simple.  There are also
constraints on how many syllables can be in a line.)

\section*{Programming Section (40 points)}

Look at the file \texttt{limerick.py} in the hw1 folder.  Your job is to fill
in the missing functions in that file so that it does what it’s
supposed to do.
\begin{itemize}
\item {\bf rhyme}: detect whether two words rhyme or not
\item {\bf limerick}: given a candidate limerick, return whether it meets the
constraint or not.
\end{itemize}
More requirements / information appear in the source files.

\clearpage

\noindent \textbf{Notes}:
\begin{itemize}
\item {\bf How do I separate words from a string of text?}

Use the \texttt{word\_tokenize} function.

\item {\bf What if a word isn’t in the pronouncing dictionary?}

Assume it doesn’t rhyme with anything and only has one syllable.

\item {\bf How ``hardened'' should the code be?}

It should handle ASCII text with punctuation and whitespace in upper or lower case.

\item {\bf What if a word has multiple pronunciations?}

If a word like “fire” has multiple pronunciations, then you should say
that it rhymes with another word if any of the pronunciations rhymes.

\item {\bf What if a word starts with a vowel?}

Then it has no initial consonant, and then the entire word should be a
suffix of the other word.
\end{itemize}
% section problem_3 (end)

\section*{Extra Credit}

Extra Credit (create new functions for these features; don’t put them
in the required functions that will be run by the autograder):
\begin{itemize}
\item[(up to 2 points)] Create a new function called
  \texttt{apostrophe\_tokenize} that handles apostrophes in words correctly so
  that ``can’t'' would rhyme with ``pant''.
\item[(up to 5 points)] Make reasonable guesses about the number of syllables in unknown words in a function called \texttt{guess\_syllables}.
\item[(up to 5 points)] Compose a funny original limerick about
  computational linguistics, natural language processing, or machine
  learning (add it to your submission as \texttt{limerick.txt}).
\end{itemize}
Add extra credit code as functions to the \texttt{LimerickDetector}
class, but don't interfere with required functionality.

\end{document}
